\chapter{Thesis content and overview}\label{overview}
The central theme of this Ph.D. has been the development and application of methods to delineate bacterial and viral diversity from metagenomics. Successful organization of bacteria and viruses into ecological species hubs allows analysis of their interactions and the implications on the metazoan host like humans. The majority of microbiome studies that explored human gut biodiversity and its implications on human health have focused primarily on the bacterial constituents. The reason for this was twofold: (1) the virome (the collective of viruses in the environment) was mostly explored in viral-enriched metagenomic samples with \textit{in vitro} isolated viral fractions and (2) the feasibility of exploring the virome from samples without viral preprocessing was barely described and problematic due to bacterial contamination. However, the growing wave of bulk/non-enriched metagenomic samples collected from the human gut, soil and marine environments is continuing to dwarf the number of collected viral-enriched metagenomic samples, which corresponds to metaviromes. Thus, there is a profound need for standardized methods to extract and explore the virome from bulk metagenomics due to their influence in the environment.

\medskip
\noindent
In \textbf{Chapter 1}, I provide a brief history on the discovery of bacterial and viral diversity in metagenomics and its current state and challenges. In addition, I outline a concise review on virus biology and dynamics with bacteria and altogether the possible implications to metazoan hosts like humans. In \textbf{Chapter 2}, I describe the bioinformatic methods and concepts with a focus on genome binning, genome annotation, machine learning and genome-driven ways to connect bacteria and viruses. In \textbf{Chapter 3}, I list and expand on the research objectives pursued during this Ph.D project. In \textbf{Chapter 4}, I describe the three major studies included in this thesis related to metagenomics binning of bacteria and viruses, and a study on the gut virome in humans with extreme longevity. In \textbf{Chapter 5}, I summarize and discuss the major results of the three studies included. In \textbf{Chapter 6}, I discuss future perspectives and suggestions for improved bacterial and virome analysis in metagenomics. \textbf{Chapter 7} contains ethical and legal permits and approvals required for the clinical and animal studies. \textbf{Chapter 8} includes the manuscripts included in this dissertation.
