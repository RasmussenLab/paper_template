\chapter{Dansk resumé}

Det menneskelige tarmmikrobiom er vært for adskillige grupper af mikroorganismer, herunder bakterier, arkæer, eukaryoter og virus. Tarmbakterier er den mest undersøgte af disse og er anerkendt for dets rolle i tarmmetabolismen og immunsystemets udvikling, begge af stor betydning for den menneskelige vært. Desuden udgør bakterier en stor del af den genetiske arvemasse i tarmen som understreger deres funktionelle betydning i økosystemet. En mindre undersøgt gruppe i tarmens økosystem er bakterieinficerende virus, også kendt som bakteriofager. Bakteriofager er anslået til at udligne eller overstige antallet af bakterier i tarmen. Givet deres destruktive interaktion med bakterier, kan de have en stor betydning for bakteriebalancen under raske og sygdomsbetingede tilstande. 
I denne afhandling udforsker vi algoritmer og computerbaserede metoder med etablerede metoder fra kunstig intelligens og maskinlæring til at opdage og karakterisere tarmens biodiversitet, herunder bakterier og vira fra menneskelige tarmmikrobiomer. I den \textbf{første} artikel præsenterer vi VAMB, en computerbaseret algoritme til metagenomisk binning, hvilket kan oversættes til gruppering. VAMB er baseret på en variational autoencoder til at repræsentere metagenomiske sekvenser som nemmere kan splittes og grupperes til samlede bakterielle genomer. I artiklen har vi demonstreret at VAMB er bedre end tilsvarende værktøjer til at gruppere syntetiske metagenomer og genskabe bakteriestammer. Ligeledes har vi anvendt VAMB på et stort metagenomisk datasæt baseret på 1000 humane fæces prøver med næsten 6 millioner sekvenser og vist at VAMB kan genskabe bakteriestammer med høj fylogenetisk præcision. I den \textbf{anden} artikel beskriver vi, hvordan VAMB kan bruges til gruppering af tarmens virus genomer i en metode vi har kaldt PHAMB. Binning af virus genomer er betinget af andre udfordringer end bakterielle genomer, da deres genomer typisk er mindre og mere fragmenterede. Selvom VAMB ikke oprindeligt blev designet specifikt for virus, har vi vist at det kan anvendes til at genskabe både bakterielle og virale genomer parallelt, hvilket muliggør undersøgelse af begge biologiske domæner i metagenomiske datasæt. Desuden etablerer vi at at binning forbedrer den totale rekonstruktion og kvalitet af virusgenomer på tværs af tre forskellige datasæt. I den \textbf{tredje} artikel anvender vi begge metoder (VAMB og PHAMB) til at karakterisere og undersøge bakterielle og virale populationer i tarmmikrobiomer fra hundredårige samt yngre kontrolgrupper. I studiet opdager vi først og fremmest ny viral diversitet der potentielt kan fremme menneskets levetid. Desuden, observerer vi at hundredåriges tarmmikrobiom huser et rigt og mangfoldigt virus system der interagerer med gavnlige bakterielle populationer. Det siges at bakteriofager kan bidrage med ekstra funktionel arvemasse til bakterierne de inficerer. Vi opdagede i forlængelse af denne hypotese at bakteriofager integreret i bakterier fra hundredårige bidrager med enzymer der faciliterer vigtige trin i bakterielle metaboliske systemer relateret til omdannelsen af sulfat til sulfid og methionin til homocystein. Tilsammen viste vi at hundredåriges tarmmikrobiom har et øget metabolisk potentiale for omdannelse af sulfat til sulfid, hvilket kan have stor betydning, da en øget mængde af svovlbrinte i tarmen kan understøtte tarmens integritet og resistens over for patogener.
Afhandlingen beskriver computerbaserede metoder til at organisere bakterier og virus arter i det menneskelige tarmmikrobiom og en dissektion af deres potentielle indvirkning på den menneskelige vært. Indtil fremtidige sekventeringsteknologier bliver omkostningseffektive, nøjagtige og gradvist erstatter nuværende sekventeringsmetoder til at studere metagenomiske prøver, vil værktøjer baseret på binning være nødvendige for at etablere bakteriel og viral diversitet samt forstå deres indviklede dynamik.